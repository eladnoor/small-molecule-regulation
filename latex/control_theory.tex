\documentclass[12pt,a4paper]{article}
\usepackage[utf8]{inputenc}
\usepackage[english]{babel}
\usepackage{amsmath}
\usepackage{amsfonts}
\usepackage{amssymb}
\usepackage{graphicx}
\usepackage{pgfplots}
\pgfplotsset{compat=1.13}
\begin{document}
\section{Control Theory}
\subsection{Single-substrate Control}
Consider an enzyme-catalyzed reaction, described by a one-substrate irreversible Michaelis-Menten kinetic rate law:
\begin{eqnarray}
    v &=& V_{max} \cdot \frac{s}{K_M + s}
\end{eqnarray}
where $s$ is the substrate concentrations (in units of molar), and $K_M$ is the Michaelis-Menten coefficient (also in molar).

The scaled elasticities for $s$ and $i$ are given by:
\begin{eqnarray}
    \epsilon_s^v &=& \frac{\partial \ln(v)}{\partial \ln(s)} = \frac{\partial v}{\partial s}\cdot\frac{s}{v} = V_{max} \cdot \frac{K_M + s - s}{(K_M + s)^2} \cdot \frac{s}{v} \nonumber \\
    &=& \frac{K_M}{(K_M + s)^2} \cdot (K_M + s) = \frac{K_M}{K_M + s} = 1 - \frac{s}{K_M + s}
\end{eqnarray}

\subsection{Non-competitive Inhibition}
Now, we add a cooperative non-competitive inhibitor $I$, with a Hill coefficient $h$, described by the rate law:
\begin{eqnarray}
    v &=& V_{max} \cdot \frac{s}{K_M + s} \cdot \left(1 - \frac{i^h}{K_I^h + i^h}\right) =
    V_{max} \cdot \frac{s}{K_M + s} \cdot \frac{K_I^h}{K_I^h + i^h}
\end{eqnarray}
where $i$ is the concentration of the inhibitor.

The following plot shows the response of $v$ to the concentration $i$ in log-log scale, when the Hill coefficient is $h = 2$. For simplicity, we assume that $\frac{s}{K_M + s} = 1$.

\begin{tikzpicture}
	\begin{loglogaxis}[width=200pt,axis x line=bottom, axis y line=left, tick align=outside, domain=1e-2:1e2, xlabel=$i/K_I$, ylabel=$v/V_{max}$]
		\addplot+[mark=none,smooth] (\x,{(1 - (\x^2/(1+\x^2))});
	\end{loglogaxis}
\end{tikzpicture}

In this case, the elasticity with regards to the inhibitor concentration would be:
\begin{eqnarray}
    \epsilon_i^v &=& \frac{\partial \ln(v)}{\partial \ln(i)} = \frac{\partial v}{\partial i}\cdot\frac{i}{v} = V_{max} ~ \frac{s ~ K_I^h}{K_M + s} ~ \frac{- h ~ i^{h-1}}{(K_I^h + i^h)^2} ~ \frac{i}{v} \nonumber \\
    &=& -\frac{h ~ i^h ~ (K_I^h + i^h)}{(K_I^h + i^h)^2} = -h ~ \frac{i^h}{K_I^h + i^h}~.
\end{eqnarray}

Plotting the elasticity as a function of $i$, we see that it is a monotonically decreasing negative function:

\begin{tikzpicture}
	\begin{semilogxaxis}[width=200pt,axis x line=bottom, axis y line=left, tick align=outside, domain=1e-2:1e2, xlabel=$i/K_I$, ylabel=$\epsilon_i^v$]
		\addplot+[mark=none,smooth] (\x,{-2 * (\x^2/(1+\x^2))});
	\end{semilogxaxis}
\end{tikzpicture}

This means that substrates have the most control ($\epsilon_s^v \rightarrow 1$) when they are much below saturation ($s \ll K_M$) while inhibitors have the most control ($\epsilon_i^v \rightarrow -h$) when they are saturated ($i \gg K_I$).


\subsection{Competitive Inhibition}
Described by the rate law:
\begin{eqnarray}
    v &=& V_{max} \cdot \frac{s}{K_M \left(1 + \frac{i^h}{K_I^h}\right) + s}
\end{eqnarray}
Solving for the elasticity:
\begin{eqnarray}
    \epsilon_i^v &=& \frac{\partial v}{\partial i}\cdot\frac{i}{v} = V_{max} \cdot s \cdot \frac{-K_M ~ K_I^{-h} ~ h~i^{h-1}}{\left(K_M \left(1 + \frac{i^h}{K_I^h}\right) + s\right)^2} \cdot \frac{i}{v} \nonumber \\
    &=& -h~\frac{K_M ~ K_I^{-h} ~ i^h}{K_M \left(1 + \frac{i^h}{K_I^h}\right) + s} =
        -h~\frac{i^h}{K_I^h ~ (1 + s / K_M) + i^h} \nonumber \\
    &=& -h~\frac{i^h}{K_E^h + i^h}
\end{eqnarray}
where we define $K_E \equiv K_I \sqrt[h]{1 + s/K_M}$, the effective inhibition constant. It is thus obvious, that the competitive inhibition case is equivalent to the non-competitive case, with a shift to the effective $K_I$ which depends on the substrate saturation level.


\end{document}